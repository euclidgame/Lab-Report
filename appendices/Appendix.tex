%!TEX root = ../Report.tex

\chapter{File Structure of \\the Project}
\label{app:sus}
\kaishu
\hspace*{5mm}本附录用于展示实验目录的文件结构。
下面的结构图展示了项目文件中的关键文件,以及他们之间的所属关系。
项目名称为\textbf{\sffamily{Naive\_cmd}},意味较为简陋、不够成熟的命令行。\\
\hspace*{5mm}文件的上下级关系表示某个文件属于某一个目录,
或者上级模块调用了下级模块,即调用与被调用的关系。
\\*[7cm]
%\documentclass{minimal}
\usetikzlibrary{trees}

\tikzstyle{every node}=[draw=black,thick,anchor=west]
\tikzstyle{selected}=[draw=red,fill=red!30]
\tikzstyle{optional}=[dashed,fill=gray!50]
\begin{tikzpicture}[%
  grow via three points={one child at (0.5,-0.7) and
  two children at (0.5,-0.7) and (0.5,-1.4)},
  edge from parent path={[->](\tikzparentnode.south) |- (\tikzchildnode.west)}]
  \node {Naive\_cmd}
  child { node[selected]{Naive\_cmd.v}}
    child { node [optional]{clkgen.v}}
    child {node {HEX and LED}
     child {node {Hexdisplay.v}}
     child {node {LEDdisplay.v}}
    }
      child [missing] {}
    child [missing] {}
    child { node {Keyboard}
      child { node {Keyboard.v}
        child { node {ps2\_keyboard.v}}
        child { node {word.v} }
        child { node[optional] {Ascii\_rom}
        }
      }
    }					
    child [missing] {}
    child [missing] {}
    child { node {VGA}
      child {node {VGA.v}
        child {node {VGA\_ram.v}}
        child {node {ROM\_show.v}}
      }
    }				
    child [missing] {}				
    child [missing] {}	
    child { node {Audio}
        child {node {Audio.v}
          child {node {一些控制模块}}
          child {node [optional]{songs}}
        }
    }
    child [missing] {}				
    child [missing] {}
    child [missing] {}
   % child {node {HEX and LED}}
    %child [missing] {}	
    %child [missing] {}
    child { 
          node [selected] {CPU\_overall}
          child {node {ALU.v}}
          child { 
            node [selected]{CPU\_overall.v}
            child {node {Registers.v}}
            child {node {data\_ram.v}}
            child {node {ALU\_part.v}}
            child {node {Decode.v}
              child {node {De\_Helper.v}}
              child {node {program\_counter.v}}
              child {node {Fetch\_code.v}
                child{node [optional]{instructions}}
              }
            }
          }
    };
    
\end{tikzpicture}
%==============================================================================================================================================

\chapter{Scripts of Special \\Purpose}
%\large
\songti
\hspace*{5mm}因为每次都要打开\file{Mars}先编译再导出机器码较为繁琐,所以我们提供了一个
从\file{Mars}中获取机器码的脚本,如下所示:
\begin{lstlisting}[style=seminar, caption = {提取代码脚本}]
import mars.*;
import java.util.*;

public class MarsCompiler {
	public static void main(String... args) throws Exception {
		if (args.length != 1) {
			System.err.println("Usage: java MarsCompiler input");
			System.exit(1);
			}

		Globals.initialize(false);
		MIPSprogram program = new MIPSprogram();
		program.readSource(args[0]);
			
		ErrorList errors = null;
			
		try {
			program.tokenize();
			errors = program.assemble(new ArrayList(Arrays.asList(program)), true, true);
			}
		catch (ProcessingException e) {
			errors = e.errors();
			}
			
		if (errors.errorsOccurred() || errors.warningsOccurred()) {
			for (ErrorMessage em : (ArrayList<ErrorMessage>)errors.getErrorMessages()) {
				System.err.println(String.format("[%s] %s@%d:%d %s",
				em.isWarning() ? "WRN" : "ERR",
				em.getFilename(), em.getLine(), em.getPosition(),
				em.getMessage()));
				}
			System.exit(2);
		}
	
		for (ProgramStatement ps : (ArrayList<ProgramStatement>)program.getMachineList())
		System.out.println(String.format("%08x", ps.getBinaryStatement()));
		}
}
\end{lstlisting}

这里运用了\file{Mars}作为\code{java}语言的特点,利用了本身固有的类来导出机器码。

只需要在命令行执行以下的指令:
\begin{lstlisting}[style = seminar]
javac --class-path MARS4_5.jar -Xlint:unchecked MarsCompiler.java
java -cp MARS4_5.jar MarsCompiler overall.asm > overall.txt
\end{lstlisting}
即可将机器码导出到指定的文件中。以上为\file{Macbook + iTerm2 + zsh}下的指令,
\file{Windows}下可能有区别。