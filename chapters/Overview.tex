%!TEX root = ../Report.tex
\chapter{Overview}
\kaishu
\hspace*{9mm}在本次实验中,我们利用Quartus和DE-10开发板,完成了5级流水线CPU的设计;
在此基础上,结合输入输出,利用CPU运行一些简单的软件,实现了一个基础的
命令行程序。

我们将对软件和硬件部分分别进行介绍,并且在附录A和B中分别给出了项目目录结构以及使用的一个脚本。

以下是实现的功能概览。
\section{硬件部分}
\songti
\begin{enumerate}
    \item \emph{实现了五级流水线CPU}。运用流水线CPU的基本原理,并在时序上做了一些
改动,实现了\coloremph{取值-译码-执行-访存-写回};
    \item 实现了ALU、仿存和跳转/分支的基本指令;
    \item 键盘、VGA和音频之间的\coloremph{协调工作};
    \item 在数据RAM中对数据段,输入输出段以及栈空间进行合理划分;
    \item 待补充
\end{enumerate}
\section{输入输出部分}
\begin{enumerate}
    \item 利用键盘向CPU传送单个字符,以及shift或caps键与单个字符组合的\coloremph{大写/符号字符};
    \item 在屏幕上显示CPU计算出的需要输出的字符,实现了显示光标,区分颜色以及屏幕上移等功能;
    \item 实现了CPU控制音频模块播放不同的音乐;
\end{enumerate}

\section{软件部分}
\begin{enumerate}
    \item 通过MIPS汇编编写并生成机器码作为指令代码段文件;
    \item 实现了与硬件接口之间的\coloremph{交互},轮询接口状态,从键盘缓冲器中读取字符并判断跳转至命令分析阶段;
    \item 命令分析阶段完成了字符串的匹配并跳转执行相关子程序,执行结果返回硬件;
    \item 子程序包括:
    \item[\color{yellow}$\star$]输入hello: 显示HELLO艺术字;
    \item[\color{yellow}$\star$]输入help: 显示命令提示信息;
    \item[\color{yellow}$\star$]输入fib: 输入十六进制数字$n$,显示屏显示第$n$个Fibonacci数的十六进制形式;
    \item[\color{yellow}$\star$]输入piano: 输入数字$i\in [1,3]$,播放第$i$首歌曲;
    \item[\color{yellow}$\star$]输入display: 输入十六进制数字,七段数码管上显示对应数字;
    \item[\color{yellow}$\star$]输入LEDon: 输入数字$i\in [0,9]$,打开第$i$个LED;
                            \\输入LEDoff: 输入数字$i\in [0,9]$,关闭第$i$个LED;
    \item[\color{yellow}$\star$]输入restart: 重新启动并恢复初始状态;
    \item[\color{yellow}$\star$]输入未知命令: 显示未知命令提示Unknown Command;
\end{enumerate}